\documentclass[10pt, letterpaper, notitlepage]{article}
\usepackage[letterpaper]{geometry}
\usepackage{setspace}
\usepackage{hyperref}
\usepackage{fancyhdr}
\usepackage{amsmath}
\usepackage{amssymb}
\usepackage[makeroom]{cancel}
\usepackage{graphicx}
\usepackage{tikz}
\usepackage{caption,subcaption}
\usepackage{natbib}

\setlength{\headheight}{15.2pt}
\pagestyle{fancy}
\lhead[ASTRUN3646]{ASTRUN3646}
\rhead[Jake Lee]{Jake Lee}

\setlength{\parindent}{0cm}

\begin{document}

\title{Automated Discovery of Unusual Spectra Observed in Galaxies}
\author{Jake Lee, jhl2195}
\maketitle

\section{Introduction}

The SDSS Mapping Nearby Galaxies at APO (MaNGA) survey is a unique dataset of spectral
measurements across the face of about 10,000 nearby galaxies \citep{2015ApJ...798....7B}.
The 3-D data reduction pipeline (DRP) product of each galaxy contains over a thousand spectra,
organized into row-stacked spectra and data cubes. However, due to the high
volume of data contained in this dataset, it can be difficult to perform exploratory work
to identify unusual or novel observations. This process is especially important for the
scientific discovery process, which relies on unexpected observations to revise current
knowledge or overturn existing theories \citep{kuhn2012structure}.\\

This work seeks to facilitate discovery in this dataset by automating the process of
prioritizing unusual or novel data. Through automation, unexpected phenomena, artifacts
due to errors, and groundbreaking discoveries can be quickly prioritized for more efficient
analysis by domain experts. The method I implement, DEMUD, is unique in its ability to
generate interpretable explanations for each of its prioritizations \citep{wagstaff2013guiding}.
These explanations provide transparency for the model's decisions, and increase users'
trust and receptiveness to DEMUD's results.

\section{Method}

\subsection{A Brief Description of DEMUD}

DEMUD is a machine learning algorithm that attempts to prioritize, or ``select'', items
of relative high interest or novelty. It does not require any prior training data or
supervision. Given a set of data, DEMUD will iteratively build a model to represent the
data it has seen so far, then use that model to determine the next novel item to be selected.
For a full technical description of the algorithm, refer to \cite{wagstaff2013guiding}.\\

The first item \textit{selection} 0 is selected as a baseline. It can be selected randomly, or it can be the
first item. The method utilized here is to model the entire
dataset first, then to select the most anomalous data point to start with.\\

After selecting the first item, DEMUD computes an SVD model using only the first item. Then, it attempts
to generate \textit{reconstructions} of each remaining data point using this model (at this stage, it has only
seen selection 0, so it simply generates selection 0 for all data points). Then, it selects the data point
most different from its reconstruction - this is now selection 1.\\

Selection 1 is the data point most different from selection 0. Its reconstruction is the model's best
attempt at recreating it, knowing only information from selection 0. Its \textit{residual}, the difference
between the selection and the reconstruction, shows what DEMUD thought was interesting about selection 1 compared
to everything seen before.\\

This continues for the next $n$ selections. For example, Selection 5 is the data point most different from
selections 0 through 4. Its reconstruction is the model's best attempt at recreating it, knowing only
information from selections 0 through 4. Its residual, the difference between the selection and the reconstruction,
shows what DEMUD thought was interesting about selection 5 compared to all selections before it.\\

The residual is the \textit{explanation} DEMUD is providing for its novel item prioritization, since it
represents all of the information that DEMUD could not generate with its prior knowledge.

\subsection{Using SDSS data with DEMUD}

I decided to apply DEMUD to the spectra across the face of one galaxy. Applying DEMUD across galaxies proved
to be a challenge due to data dimensionality. Additionally, I sought to take advantage of MaNGA's main
contribution of spectra across the face of each galaxy.\\

I used the Marvin python library to access MaNGA data \citep{2018arXiv181203833C}. For each galaxy,
I extracted each available spaxel coordinate and 1-D spectra (flux) generated by the 3-D DRP into a CSV.
DEMUD has already been configured to ingest a CSV of 1-D vectors, and generates a list of selected items
(ID'd as spaxel coordinates) as well as three separate CSVs for the ``select'', ``reconstruction'', and ``residual''
vectors. I then generated visualizations for these CSVs in a separate script.\\

Filtering out known bad data proved to be a challenge. I wanted to
filter out all known bad data before discovery so that the DEMUD
output would be more scientifically meaningful (instead of just
identifying errors and artifacts). The dataset provides many different
bitmasks to mask out data deemed unusable for science. However, some of
these bitmasks proved to be too aggressive; for example, the 3D bitmask
for the flux datacube marks every spaxel unusable for some wavelength.
The method I adopted was to use the bitmask of gflux emline 2D
products, which does eliminate all spectra with errors, but is still
too aggressive.\\

All code for my CSV extraction, DEMUD, and visualization scripts, as well as documentation detailing their usage,
will be available in the following repositories:
\begin{itemize}
	\item \url{https://github.com/wkiri/DEMUD}
	\item \url{https://github.com/jakehlee/astro_final}
\end{itemize}

\section{Results}

\subsection{Visualization Description}
Before interpreting the results, I will describe the visualizations generated by my method.
\begin{itemize}
	\item The title starts with the plate IFU designation---this is used instead of MaNGA-ID to avoid issues with multiple observations. It also specifies which DEMUD selection it was.
	\item The top left image is the optical image of the galaxy with the hexagonal overlay showing IFU coverage.
	\item The top right figure shows the location of this specific spaxel.
	\begin{itemize}
		\item The title specifies the coordinates of the spaxel.
		\item The outer gray area masks the area not covered by the IFU.
		\item The inner gray area with white crosses masks the area deemed bad data or high SNR by this specific observation - marked by the bit DONOTUSE.
		\item The innermost green area marks the region of spaxels with valid/reliable data.
		\item The black square dot, if present, marks the location of this spaxel within the green region of valid data.
		\item The red square dot, if present, marks the location of this spaxel within the gray and white cross region of DONOTUSE data.
	\end{itemize}
	\item The middle figure plots the selected spectra (in green) as well as the model's best attempt to reconstruct it,
	given its previous selections (in blue). A light red highlights the area in between, which is the residual. Note the log-y axis.
	\item The bottom figure plots the residual spectra (in red). Note the linear-y axis.
\end{itemize}

\section{Conclusion and Future Work}

This project demonstrates that DEMUD can be successfully implemented on spectral data from SDSS MaNGA. Further improvements
are necessary with visualization and data pre-filtering, and more experiments need to be run on more galaxies for further
insights.

\newpage
\bibliographystyle{apalike}
\bibliography{mybib.bib}

\end{document}